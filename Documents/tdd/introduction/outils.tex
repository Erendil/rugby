\section{Outils}

Pour notre projet, nous utiliserons plusieurs outils :\\

\begin{tabular}{|c|c|c|}
	\hline
		\rowcolor{table_header_color} 
		\textbf{Nom} & \textbf{Version} & \textbf{Utilité} \\
	\hline
	\multicolumn{3}{c}{}\\
	\hline
	\multicolumn{3}{|c|}{\textbf{Moteur de jeu}}\\
	\hline
		Unity & 4.0 & L'outil principal de développement pour le jeu\\
	\hline
	\multicolumn{3}{c}{}\\
	\hline
	\multicolumn{3}{|c|}{\textbf{Programmation}}\\
	\hline
		Visual Studio & & \multirow{2}*{Editeur de texte pour les scripts} \\
	\cline{1-2}
		MonoDevelop & & \\
	\hline	
	\multicolumn{3}{c}{}\\
	\hline
	\multicolumn{3}{|c|}{\textbf{Graphismes}}\\	
	\hline
		Maya & & Modelisation, animation, set-up \\
	\hline
		Z-Brush & & Sculpt, texturing \\
	\hline
		Photoshop & & painting, texturing\\	
	\hline
		Paint.NET & & Dessin  \\	
	\hline
		Illustrator & & Dessin vectoriel, design d'interface \\
	\hline
	\multicolumn{3}{c}{}\\
	\hline
	\multicolumn{3}{|c|}{\textbf{Documentation}}\\
	\hline
		Office & & Ensemble de logiciels de création de document \\
	\hline
		\LaTeX & & Langage de création de documents\\
	\hline
	\multicolumn{3}{c}{}\\
	\hline
	\multicolumn{3}{|c|}{\textbf{Espace de stockage}}\\
	\hline
		git & & gestionnaire de version en ligne\\
	\hline
		dropbox & & système de cloud pour les versions clés.\\
	\hline
\end{tabular}