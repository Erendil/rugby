\section{Intelligence Artificielle}

%Dans un premier temps, nous partons sur une simple "state machine" (un graphe d'actions).
%Nous sommes en train de nous documenter sur la manière dont les IA sont faites dans les jeux d'équipe qui existent.
%Nous sommes donc en train de lire plusieurs documents que nous trouvons sur internet tels que ce mémoire : \href{http://www.drieu.org/docs/memoire.pdf}{http://www.drieu.org/docs/memoire.pdf} qui semble bien.

Dans un premier temps nous étions partis sur un automate fini dont les états pouvaient soit appeler un sous-automate, soit générer des ordres qui seraient gérés par les state-machines des unités; les états étant reliés par des 'conditions' (delegates).
Cette solution avait pour avantage de pouvoir être gérée par un 'éditeur'. Mais cela impliquerait que le résultat soit stocké dans un fichier, être chargé (parsé), de devoir créer l'éditeur (surement un programme en C avec la \ac{sdl}) qui crée le fichier. 
Bien que cela ne devrait, en soi, pas être compliqué à créer, et encore moins à utiliser (même pour les \ac{gder}), cela risquerait de prendre vraiment trop de temps à fabriquer.. \\

Nous allons donc créer la state-machine pour chaque unité, nous allons faire également un système agent mais sans éditeur, ils auront un code fixe (on reprendra notre ancienne idée que si nous sommes en avance).
